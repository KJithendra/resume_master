\documentclass[a4paper,11pt]{article}

\usepackage{times} % Font Package
\usepackage[top=0.5in, bottom=0.5in, left=0.35in, right=0.65in]{geometry}
\usepackage{graphicx} %Use of photos
\usepackage{multicol}

\usepackage{array}

\usepackage[T1]{fontenc}
\usepackage[utf8]{inputenc} % input encoding

\usepackage{nopageno}

\setlength{\tabcolsep}{0in}
\newcommand{\isep}{-2 pt}
\newcommand{\lsep}{-0.5cm}
\newcommand{\psep}{-0.6cm}
\newcommand{\spsep}{-0.75cm}

\newcommand{\tfspace}{\hspace{4.5mm}}

\newcommand{\resheading}[1]{{\large {\begin{minipage}{1\textwidth}{\uppercase{ \textbf{#1}}}\end{minipage}}}}

\renewcommand{\labelitemii}{$\circ$}
%\newcommand{\resheading}[1]{{\Large {\begin{minipage}{0.975\textwidth}{\scshape{\textbf{#1 \vphantom{p\^{E}}}}}\end{minipage}}}}

\usepackage{hyperref}
\hypersetup{
    colorlinks=true,
    linkcolor=cyan,
    filecolor=magenta,      
    urlcolor=blue,
}


\begin{document}

%Personal Details and Institute logo
\begin{center}
\LARGE { \uppercase {\textbf{K Jithendra}}}\\
\large \textbf{{RTL Design Engineer at Intel | IIT Madras | NIT Puducherry | Ex-Mphasis | Exp: VLSI-4.75Y, IT-1Y}}\\
\large {+91 9489551597 | jithendra.kothakota@gmail.com | \href{https://www.linkedin.com/in/kjithendra/}{https://www.linkedin.com/in/kjithendra/}} \\
\end{center}


%\begin{minipage}{0.5\linewidth}
%\Large {RTL Design Engineer at Intel}\\
%\Large {IIT Madras, NIT Puducherry, Ex-Mphasis}\\
%\end{minipage}
%\begin{minipage}{0.5\linewidth}
%%\\
%\Large {+91 9489551597}\\
%\Large {jithendra.kothakota@gmail.com}\\
%\Large {\href{https://www.linkedin.com/in/kjithendra/}{https://www.linkedin.com/in/kjithendra/}}\\
%% \frame{\includegraphics[width=30mm]{jithendra_k.png}}
%\end{minipage}\\

%\begin{minipage}{0.70\linewidth}
%\LARGE \uppercase {\textbf{K Jithendra}}\\
%\Large \uppercase{{RTL Design Engineer, Intel Corporation}}\\\\
%\Large {IIT Madras, NIT Puducherry, Ex-Mphasis}\\
%\Large {+91 9489551597 | jithendra.kothakota@gmail.com}\\
%\Large {\href{https://www.linkedin.com/in/kjithendra/}{https://www.linkedin.com/in/kjithendra/}}\\
%\end{minipage}
%\begin{minipage}{0.27\linewidth}
%\hfill
%\frame{\includegraphics[width=30mm]{jithendra_k.png}}
%\end{minipage}\\

\resheading{\textbf{Professional Experience \hspace*{6cm} VLSI - 4.75 Years, IT - 1 Year}}\\[\lsep]
\setlength{\unitlength}{5cm}
\put(0.11,-0.04){\line(1,0){3.6}}\\[-0.6cm]
\begin{itemize}
	\item \textbf{RTL Design Engineer - Power Management(PM) Unit, at Intel Corporation \hfill February 2022 - Present} \\
	\emph{Synopsis VCS MX, Verdi, SpyGlass for Lint, CDC, RDC,  \hfill Verilog, System Verilog, Perl, Make files,\\and DFT, MS Excel\hspace*{80mm} On Chip SB Protocol , Tcsh scripting, Git} \\[\spsep]
		\begin{itemize} \itemsep \isep
			\item \textbf{Ownership of Dispatcher controller}: Sequence and send power related work points to various IPs present in the GPU.
				\begin{itemize} \itemsep \isep
					\item \textbf{Designed Dispatcher FSM} that sends work points to IP driver sequencers parallelly and with standardized commands so that the power management actions on the IPs are made more modular. 
					\item Implemented \textbf{unit level tests} for this design to check basic functional correctness.
					\item Designed DVFS workpoint handling in Dispatcher along with firmware override capability. 
					\item \textbf{Fixed identified bugs} and \textbf{supported validation team} throughout the project life cycle.
					\item Analyzed equivalence checks for a functional improvement in the dispatcher module to decide whether to proceed with \textbf{ECO} (Engineering change Order). 
				\end{itemize}
			% \item Owned PM Unit's Dispatcher module that sequences and sends power related work points to various domains and partitions in the GPU. Implemented DVFS handling with firmware override capability. Fixed bugs and supported validation team throughout the project life cycle.
			% \item Implemented Modular IP driver in Dispatcher which makes the addition or removal of an IP easy and the system to be more modular. This make the work point actions by various IPs more parallel and independent.
			\item \textbf{Ownership of PFET Controller and Fuse Store units}
				\begin{itemize} \itemsep \isep
					\item \textbf{Integrated} FSM IP into PFET controller unit. Updated wrapper module as per the new requirements.
					\item \textbf{Added} new fuses required for the latest project into fuse store unit.
					\item \textbf{Supported} the IP integration team with dangles and ijtag integration issues. Supported the teams consuming the fuses with fuse details, mmio details, and debugs. % Include caliber later
					\item Learned and improved skills to work on IPs for multiple projects parallelly.% on Graphics and Media IPs for various projects parallelly.				
				\end{itemize}
			\item \textbf{Parameterization of features in PM Unit}: Addition or Removal of features from the IP using parameters
			\begin{itemize} \itemsep \isep
				\item Updated necessary changes in the \textbf{design flow}. \textbf{Collaborated} with Tools and Flows team to make the necessary changes in the used tools and its configurations.
				\item \textbf{Demonstrated POC} by generating two different PM Units that have differences in their features.
				\item \textbf{Integrated} into the \textbf{latest project} to speed up the design timeline of the project.
			\end{itemize}
			% \item Implemented Parameterization in the PM Unit repository to make it a central repository from which multiple PM units, with different features, can be generated. Made necessary changes required in the RTL design flow, collaborated with Tools and Flows team to make the necessary changes in the tools used. Demonstrated POC of it, made the incremental changes based on the feedback received. Introduced it in the latest project to speed up the design time-line of the project.%Implemented a central repository for the PM Unit git repository from which multiple PM Units can be generated.
			% \item Excellent team member who constantly questions the approach, provides constructive feedback to team members, keeps the workspace active and energetic with active interactions and fun events. Actively grows professional network and improves interpersonal skills.
		\end{itemize}

	\item \textbf{Architect - SoC Power Management and Reset, at Intel Corporation \hfill July 2020 - February 2022} \\
	\emph{Github, Microsoft Visio, Gitlab,  \hfill Multi markdown, UML, Python, Tcsh scripting, Git} \\[\spsep]
		\begin{itemize} \itemsep \isep
			% Write the important points with quantitative results.
			\item \textbf{Created High level Architecture specification}(HAS) for Arbitration logic between Xtensa Micro controller and TAP logic such that the requests from the micro controller are prioritized to maximize overall performance of the micro controller.
			\item \textbf{Created HAS for SVID}(Serial voltage ID) protocol related power management that is used by the firmware team to transition its source code base from Foxton assembly code to C++.
			\item HBM debug in Data center GPU debug
				\begin{itemize} \itemsep \isep
					\item \textbf{Automated} various stages of HBM emulation runs like preprocessing input files, launching emulation runs for various tests in parallel, generating the statistics such as bandwidth, latency and transactions, and generating spreadsheets with summary of the finished emulation runs.
					\item \textbf{Reduced time} required to do the above tasks by approximately 10 times.
					\item Ran various scenarios of the workloads and generated the summary for these runs.
				\end{itemize}
			% \item Automated various stages of HBM Emulation runs in Data center GPU like preprocessing input files, launching emulation runs for various tests at once in parallel, generating the statistics such as bandwidth, latency and transactions by running the corresponding scripts, and generating summaries of the finished emulation runs into spreadsheets. Reduced corresponding time required to do these tasks by  approximately 10 times. Debugged and helped to improve HBM performance by running various scenarios of work loads.		
			\item \textbf{Modelled} Adaptive clock modulation and Proportional-Integral-Derivative(PID) controller loop in python which can be used to find the best parameters that generate optimal performance.
		\end{itemize}
	
	\item \textbf{FPGA Design Engineer at Indigenous 5G Testbed, IIT Madras \hfill May 2019 - June 2020} \\
	\emph{Vivado, Vivado HLS, Xilinx Zynq UltraScale+ RFSoC, Gitlab \hfill Vivado HLS, AMBA AXI, C++, Verilog, Git} \\[\spsep]
	% \emph{Indigenous 5G Testbed is } \\[\spsep]
	\begin{itemize} \itemsep \isep
		% \item Design and implementation of 5G technologies for wireless systems and networks on integrated multi-core programmable SoCs for maximal performance. 
		\item \textbf{Designed and tested Channel Decoder} module for the Uplink Receiver of 5G NR on Xilinx Zynq UltraScale+ RFSoC FPGA.
		\item \textbf{Worked on end to end design flow} that includes design using Vivado HLS(High Level Synthesis), Verification using verilog testbenches, synthesis, place and route, and programming FPGA in Xilinx Vivado. 
		\item \textbf{Integrated} Channel decoder into Uplink Receiver and \textbf{tested} the integrated design for functional correctness.
		\item \textbf{Supported integration team} during up-revisions of the receiver.
	% \item worked on end to end VLSI design flow for Channel Decoder module like module design using HLS, verification by writing testbenches, synthesis, place and route, and testing on the Xilinx Zync FPGA.
	% \item Integrated this module into the Uplink Receiver and supported the integration team during the up-revisions of the receiver.
	\end{itemize}

\pagebreak
% %% Block comment %%%%%%%%%%%%%%%%%%%%%%%%%%%%%%%%%%%%%%%%
% \iffalse

	\item \textbf{Associate Software Engineer at Mphasis Pvt. Ltd \hfill July 2017 - July 2018} \\
	\emph{Full Stack Web development, Eclipse \hfill Java, Java servlets \& applets, Spring MVC, OOPs} \\[\spsep]
		\begin{itemize} \itemsep \isep
		% \item Worked as a team member in a project to fix vulnerabilities present in Centive, an Incentive Application based on Java Applets.
		\item \textbf{Researched and proposed solutions} to the vulnerabilities present in Centive, an incentive application based on Java applets. These proposals are accepted for implementation. Did set up the development environment as it was developed on old software stack based on Java applets.
		% \item Made regular interactions with the team members and with the topic experts from USA branch to clear the roadblocks and finish setting up the development environment.
%		\item Proposed solutions were accepted by the higher members of the team and the development environment was set up.
		\item \textbf{Developed} a basic bank web application that can be used by bank employees to carryout transactions \& process customer needs, and by customers to use internet banking facilities, as part of full stack web development training in Java programming language.
		\end{itemize}
% \fi
% %% Block comment %%%%%%%%%%%%%%%%%%%%%%%%%%%%%%%%%%%%%%%%
\end{itemize}

\resheading{\textbf{Education} }\\[\lsep]
\setlength{\unitlength}{5cm}
\put(0.11,-0.04){\line(1,0){3.6}}\\[-0.6cm]
\begin{table}[h!]
\setlength{\tabcolsep}{3.5pt}
\begin{tabular}{llll}
%\setlength{\unitlength}{5cm}
%\put(0.11,0.1){\line(1,0){3.6}}\\[-0.6cm] 
\tfspace\textbf{Program} & \textbf{Institution} & \textbf{CGPA/\%}           & \textbf{Completion Year}  \\ 
\setlength{\unitlength}{5cm}
\put(0.11,0.1){\line(1,0){3.6}}\\[-0.45cm] 
\tfspace M.Tech. (Microelectronics and VLSI Design)  & IIT Madras  & 8.85 & 2020 \\\\[-0.4cm]
\tfspace B.Tech. (Electronics and Comm. Engg.) & NIT Puducherry & 8.64 & 2017\hfill \\\\[-0.4cm]
\tfspace Intermediate & Sri Vidya Vikas Jr. College Chittoor & 97.7 & 2013\hfill \\\\[-0.4cm]
\tfspace SSC & Jawahar Navodaya Vidyalaya Chittoor & 8.4 & 2011\hfill \\\\ [-0.4cm]
\setlength{\unitlength}{5cm}
\put(0.11,0.1){\line(1,0){3.6}}\\[-0.75cm]
\end{tabular}
\end{table}

% \pagebreak
%% Block comment %%%%%%%%%%%%%%%%%%%%%%%%%%%%%%%%%%%%%%%%
\iffalse

\resheading{\textbf{Scholastic Achievements} }\\[\lsep]
\setlength{\unitlength}{5cm}
\put(0.11,-0.04){\line(1,0){3.6}}\\[-0.6cm] 
\begin{itemize}\itemsep \isep
	\item Attended India Mobile Congress(IMC), October 2019 where we exhibited Indigenous 5G Testbed Project.\hfill %Team member focussed in the design, implemention and integration of modules present in uplink receiver of physical layer in 5G cellular network.
	\item Secured All India Rank of 347 in Gate 2018 - Electronics and Communication Engineering.\hfill
	\item Participated in national round of INDO-US Robo League 2015 in Line Following event, conducted by Technophilia Systems and Robotics \& Computer Applications Institute of USA, held at IIT Bombay during AAVRITI 2015.
	\item Recipient of Central Sector Scheme of Scholarship for College and University students for the year 2013-14 and received the scholarship for 4 consecutive years from 2013 to 2017.\hfill
	\item Secured All India Rank of 14475 in JEE-MAIN 2013.\hfill
	\item Secured rank of 2724 in EAMCET 2013.\hfill
\end{itemize}

\fi
%% Block comment %%%%%%%%%%%%%%%%%%%%%%%%%%%%%%%%%%%%%%%%

\resheading{\textbf{Technical Skills}}\\[\lsep]
\setlength{\unitlength}{5cm}
\put(0.11,-0.04){\line(1,0){3.6}}\\[-0.6cm]
\begin{itemize} \itemsep \isep
  \item \textbf{HDL}\hspace{33mm}: Verilog, System Verilog, Vivado HLS (High Level Synthesis). \\[-0.55cm].
  \item \textbf{Programming languages} : Python, Perl(intermediate), tcsh scripting, Multi Markdown, C, C++, TCL(novice),\\
  \hspace*{44mm}Java(intermediate)\\[-0.55cm]%, TCL(novice). 
  \item \textbf{Software packages}\hspace{11mm}: Synopsys Verdi, VCS MX, Spyglass, Git, Jira, Vivado, Vivado HLS,  LTSpice, \\
  \hspace*{44mm}Electric, Eclipse, MATLAB, Pytorch, Spring MVC framework%, Cachegrind, CACTI, Ramulator(novice),
  \item \textbf{Other}\hspace{32mm}: AMBA AXI Protocol, Intel's On chip SB protocol
\end{itemize}

%\pagebreak
%\resheading{\textbf{Backup} }\\[\lsep]
%\setlength{\unitlength}{5cm}
%\put(0.11,-0.04){\line(1,0){3.6}}\\\\\\[-0.6cm]

\resheading{\textbf{Key Projects} }\\[\lsep]
\setlength{\unitlength}{5cm}
\put(0.11,-0.04){\line(1,0){3.6}}\\[-0.6cm]
\begin{itemize}
%% Project description template
%	\item \textbf{<Project name> \hfill <date> - <date>} \\
%	\emph{<course name>  \hfill <Skills used>} \\[\spsep]
%	\begin{itemize} \itemsep \isep
%		\item  
%		\item
%	\end{itemize}

%% Block comment %%%%%%%%%%%%%%%%%%%%%%%%%%%%%%%%%%%%%%%%
\iffalse

	\item \textbf{Employee referral program \hfill July 2021 - October 2022} \\
	\emph{Personal  \hfill Python, Selenium, Professional Networking, Digital outreach skills} \\[\spsep]
	\begin{itemize} \itemsep \isep
		\item Generated 1.5 times of my annual salary with 2.5\% of efforts.
		\item Automated repetitive tasks in the free time during the first 2 weeks of the project, there by making the main core work to be done seamless.
		\item Increased my Professional network followers by 22 times for wider reach of
		\item Increased number of views by 
		\item Digital Marketing, Search Engine Optimization
	\end{itemize}

\fi
%% Block comment %%%%%%%%%%%%%%%%%%%%%%%%%%%%%%%%%%%%%%%%

	\item \textbf{Analysis of bigLittle systolic array design using Scale-Sim \hfill January - June 2020} \\
	\emph{Systems Engineering for Deep Learning course \hfill Python, Pytorch} \\[\spsep]
	\begin{itemize} \itemsep \isep
		\item  Implemented various big little compute clusters instead of uniform symmetric compute clusters, in Python, to increase utilization of the compute resources and to increase power to performance ratio for Deep learning applications.
		\item The concept of big little architectures from CPUs are used to implement in the systolic arrays used for \textbf{Deep learning}/AI accelerators.
	\end{itemize}
	 
	 \item \textbf{Hardware accelerator for Handwritten digit recognition using MNIST database \hfill January - May 2019} \\
	\emph{Mapping Signal Processing Algorithms to DSP Architectures course \hfill C++, Vivado, Vivado HLS} \\[\spsep]
	\begin{itemize} \itemsep \isep
	\item \textbf{Implemented hardware accelerator} for trained feed forward neural network model from KANN library on Xilinx Zynq-7000 SoC, to speed up the classification of handwritten digit images from MNIST database.
%		\item Implemented a hardware accelerator to speed up the classification of the handwritten digit images from MNIST dataset.
%		\item Already trained multi layer perceptron feed forward neural network model from KANN Library is used to classify the images. This model consists of two fully connected layers and two non linearity layers.
		\item Achieved a speed improvement by a \textbf{factor of (1.8)}.
	\end{itemize}

%% Block comment %%%%%%%%%%%%%%%%%%%%%%%%%%%%%%%%%%%%%%%%
\iffalse
	
	\item \textbf{In memory compute engine for Handwritten digit recognition using MNIST database  \hfill January - May 2019} \\
	\emph{Advanced Topics in VLSI \hfill  Electric, LTSpice, MATLAB} \\[\spsep]
	\begin{itemize} \itemsep \isep
		\item Designed 4$\times$2 array of 8T SRAM Cells and associated peripheral circuitry that computes multiply and accumulate(MAC) operations for fully connected layer of trained neural network model from KANN library.
%		\item Stored the weights into the cells of the array. In compute operation, the array computes the MAC operation between input data and stored weights, then sends the result through the data bus.

	\end{itemize}
	
	\item \textbf{8 bit carry save multiplier with single stage pipeline \hfill July - November 2018} \\
	\emph{Digital IC Design \hfill  Electric, LTSpice} \\[\spsep]
	\begin{itemize} \itemsep \isep
		\item Designed schematic and layout of signed 8 bit carry save multiplier and then RC extracted netlist of Layout is used for simulations.
		\item Achieved 93\% improvement in the maximum frequency of operation of the pipelined multiplier when compared with the unpipelined multiplier.
%		\item Designed an 8 bit Carry save multiplier in transistor level schematic and layout of the same in 22nm CMOS technology, to increase the maximum frequency of operation and throughput.
%		\item Achieved a 64\% improvement in the maximum frequency of operation of the pipelined multiplier when compared with the unpipelined multiplier.
%		\item The use of Carry look ahead adder instead of ripple carry adder in the unpipelined multiplier design improved the maximum frequency of operation by 34\%.
	\end{itemize}
	
	\item \textbf{Performance evaluation and implementation of SVM Classifier for speech emotion recognition using Berlin database \hfill January - April 2017} \\
	\emph{B. Tech Project  \hfill  MATLAB} \\[\spsep]
	\begin{itemize} \itemsep \isep
		\item Trained SVM classifier using training data of 4-dimensional feature vectors. The resultant hyperplane was used as a decision boundary to classify the test data.%that separates the trainining data to a maximum possible extent.
		\item Achieved an average classification accuracy of 72.04\%.
	\end{itemize}

\fi
%% Block comment %%%%%%%%%%%%%%%%%%%%%%%%%%%%%%%%%%%%%%%%


%	\item \textbf{Web application for a bank  \hfill October 2017}  \\
%	\emph{Full Stack Web Development Training.\hfill Java, Spring MVC 4.0} \\[\spsep]
%	\begin{itemize} \itemsep \isep
%		\item Developed a bank web application that can be used by bank employees to carryout transactions \& process customer needs and by customers to use internet banking facilities.
%	\end{itemize}
\end{itemize}
\




\resheading{\textbf{Positions of Responsibility}}\\[\lsep]
\setlength{\unitlength}{5cm}
\put(0.11,-0.04){\line(1,0){3.6}}\\[-0.6cm]
\begin{itemize} \itemsep  \isep
	\item Teaching assistant - Mapping signal processing algorithms to DSP architectures course\hfill \emph{August 2019 - June 2020} \\[\spsep]
	\begin{itemize} \itemsep \isep
		\item Prepared verilog testbenches and  multiple choice questions for assignments. 
		\item Organized quiz sessions and evaluated assignment submissions and its demonstrations by the students.
		\item Demonstrated and documented assignment submission guidelines and procedure for vivado project exporting and importing.
	\end{itemize}

%% Block comment %%%%%%%%%%%%%%%%%%%%%%%%%%%%%%%%%%%%%%%%
\iffalse	
	\item Teaching assistant - Computer organization course \hfill \emph{August - November 2019} \\[\spsep]
	\begin{itemize} \itemsep \isep
		\item Mentored and evaluated students in the design and implementation of a pipelined CPU that supports RV32I Base Instruction Set of RISC-V ISA.
		\item Prepared verilog testbenches and questions for assignments.
		\item Organized and assisted lab sessions for this course.	
	\end{itemize}
%% Block comment %%%%%%%%%%%%%%%%%%%%%%%%%%%%%%%%%%%%%%%%
\fi

%% Block comment %%%%%%%%%%%%%%%%%%%%%%%%%%%%%%%%%%%%%%%%
\iffalse

	\item Teaching assistant - Mapping signal processing algorithms to architectures \hfill \emph{August - November 2019} \\[\spsep]
	\begin{itemize} \itemsep \isep
		\item Documented assignment submission instructions, like the directory structure to be used for Vivado projects, exporting a vivado project, and recreating a vivado project from the exported files. Presented these details in a session to the class and cleared doubts.
		\item Organized and assisted lab sessions for this course. Helped students with assignment related issues during these sessions.
		\item Evaluated the assignment submissions. During the assignment demonstrations, evaluated the students on the basis of level of understanding of the concepts and the code, ability to test the assignment on FPGA board etc.
		\item Prepared multiple choice questions for assignments among which some are included in the assignments.
		\item Assisted in creating the testbenches for assignments.
	\end{itemize}
	
	\item Teaching assistant - Computer organization course \hfill \emph{August - November 2019} \\[\spsep]
	\begin{itemize} \itemsep \isep
		\item Mentored and evaluated students in the design and implementation of a pipelined CPU that supports RV32I Base Instruction Set of RISC-V ISA.
		\item Assisted in creating the testbenches for assignments.
		\item Prepared multiple choice questions for assignments among which some are included in the assignments.
		\item Organized and assisted lab sessions for this course.	
	\end{itemize}
	
	\item Teaching assistant - NPTEL - Mapping signal processing algorithms to architectures \hfill \emph{August - November 2019} \\[\spsep]
	\begin{itemize} \itemsep \isep
		\item Resolved doubts that were asked in the discussion forum of this online course.
		\item Prepared multiple choice questions for assignments among which some are included in the assignments.
	\end{itemize}
	
	\item Teaching assistant - Digital systems and Lab \hfill \emph{January - May 2019} \\[\spsep]
	\begin{itemize} \itemsep \isep
		\item Coordinated weekly lab sessions and assisted students during these lab sessions. 
		\item Evaluated lab assignments and exam papers.
	\end{itemize}
	
	%\item Captain of Institute volleyball team during All India Inter NIT sports meet 2016-17 organized by NIT surathkal.
	
	%\item Mentor to NPTEL Course named Mapping signal Processing algorithms to DSP Architectures. \hfill \emph{(August-November 2019)}

%% Block comment %%%%%%%%%%%%%%%%%%%%%%%%%%%%%%%%%%%%%%%%
\fi

\end{itemize}

%\resheading{\textbf{Co-Curricular Activities}}\\[\lsep]
%\begin{itemize} \itemsep  \isep
%\item Participated in Line following event at the national round of Indo-US robo League 2015 conducted by Technophilia systems and Robotics and Computer Applications Institute of USA, held at IIT bombay during Aavriti 2015 organized by EESA IIT Bombay, on 28th and 29th march 2015.
%\item Accelero-Botix, A workshop conducted by technophilia systems in association with robotics and computer applications institute of USA held at national institute of technology Trichy on 27and 28 feb 2015.
%\item Hand gesture controlled robotics workshop organized.
% by Pragyan, The international techno- management festival of NIT tiruchirapalli held from 26 feb to 1 march 2015.
%\item Arduinotics workshop conducted by DigimindIndia pvt ltd. at Tarang 14, the annual technical symposium of ece dept held at nit pducherry on 28 feb 2014.
%\item Android application developement workshop conducted during knosys 14 on 31 jan 2014 at NIT puducherry.
%\end{itemize}


%% Block comment %%%%%%%%%%%%%%%%%%%%%%%%%%%%%%%%%%%%%%%%
\iffalse
esheading{\textbf{Relevant Coursework}}\\[\lsep]
\setlength{\unitlength}{5cm}
\put(0.11,-0.04){\line(1,0){3.6}}\\[-0.6cm]
\\\\[-0.1cm]
\hspace*{0.55cm} Systems Engineering for Deep Learning | Computer Architecture | Mapping signal processing algorithms to DSP \hspace*{0.55cm} architectures %|
\\[-0.2cm]

\fi
%% Block comment %%%%%%%%%%%%%%%%%%%%%%%%%%%%%%%%%%%%%%%%


%% Block comment %%%%%%%%%%%%%%%%%%%%%%%%%%%%%%%%%%%%%%%%
\iffalse

\resheading{\textbf{Relevant Course Work}}\\[\lsep]
\setlength{\unitlength}{5cm}
\put(0.11,-0.04){\line(1,0){3.6}}\\[-0.6cm]
\\\\[-0.1cm]
\begin{minipage}[t]{10cm}
	\begin{itemize}\itemsep \isep
		\item Computer architecture
		\item Mapping signal processing algorithms to DSP architectures
		\item Digital IC design
		\item Digital system testing and testable design
		\item Advanced topics in VLSI	
\end{itemize}
\end{minipage}
\begin{minipage}[t]{8cm}
	\begin{itemize}\itemsep \isep
		\item Pattern recognition\hfill
		\item Networks and protocols\hfill
		\item VLSI technology
		\item Semiconductor device modelling
		\item Microprocessors and microcontrollers\hfill		
\end{itemize}
\end{minipage}\\\\[-0.3cm]



\resheading{\textbf{Laboratories}}\\[\lsep]
\setlength{\unitlength}{5cm}
\put(0.11,-0.04){\line(1,0){3.6}}\\[-0.6cm]
\\\\[-0.1cm]
\begin{minipage}[t]{10cm}
	\begin{itemize}\itemsep \isep
		\item VLSI design laboratory\hfill
		\item Microprocessors and microcontrollers laboratory\hfill
\end{itemize}
\end{minipage}
\begin{minipage}[t]{8cm}
	\begin{itemize}\itemsep \isep
		\item Electronic circuits laboratory\hfill 
		\item Digital Electronics laboratory\hfill		
\end{itemize}
\end{minipage}\\\\[-0.3cm]

\fi
%% Block comment %%%%%%%%%%%%%%%%%%%%%%%%%%%%%%%%%%%%%%%%


\resheading{\textbf{Extra Curricular and co-curricular Activities}}\\[\lsep]
\setlength{\unitlength}{5cm}
\put(0.11,-0.04){\line(1,0){3.6}}\\[-0.6cm]
\begin{itemize} \itemsep  \isep
	\item Secured \textbf{All India Rank of 347} out of 1,25,000 candidates in \uppercase{Gate 2018} - Electronics and Communication Engineering exam.
	\item Won \textbf{gold medal} in Smash Wars - Volleyball tournament in Intel, held from February to March, 2023. Played as Outside hitter.
	\item Achieved \textbf{(9-16)th position}, among 260 participants, in Intel India Blitz Chess tournament, 2023.
	% \item Achieved second position in Volleyball Tournament during schroeter 2018-19, held at IIT madras.\hfill
	% \item Achieved second position in Chess competition in Annual Sports meet 2016-17, held at NIT Puducherry.\hfill
	% \item Participated and completed Sports for mental health run (5KM running competition) conducted by Shaastra Sports Tech Summit and Decathlon on October 28, 2018 in IIT Madras.\hfill
%	
%	\item Achieved first position in Freshie volleyball tournament during 2018-19		 held at IIT Madras.
%	\item Achieved First position in Carroms competition in Annual Sports meet 2016-17.
%	
%	\item Participated in Volleyball Tournament during All India Inter NIT sports meet 2016-17 organized by NIT surathkal.
%	\item Achieved second position in Chess competition at the annual sports meet 2015 of NIT Puducherry.
%	\item Achieved second position in Volleyball tournament at the annual sports meet 2015 of NIT Puducherry.
%	\item Participated in Volleyball Tournament during All India Inter NIT sports meet 2013-14 organized by NIT Warangal.
%	\item Achieved first position in Kabaddi competition held in connection with college day celebration for the year 2011-12 by Sri Vidya vikas junior college, chittoor.

	
\end{itemize}


\resheading{\textbf{Additional}}\\[\lsep]
\setlength{\unitlength}{5cm}
\put(0.11,-0.04){\line(1,0){3.6}}\\[-0.6cm]
\begin{itemize} \itemsep  \isep
	\item \textbf{Related Course work}: Systems engineering for  deep learning | Computer architecture | Mapping signal processing algorithms to DSP architectures | Advanced topics in VLSI (SRAM and eDRAM)
	\item \textbf{Hobbies}: Volleyball | Gym | Foosball | Sudoku
 	\item \textbf{VISA Sponsorship}: Need VISA sponsorship to work outside of India. I am an Indian.
\end{itemize}


%% Block comment %%%%%%%%%%%%%%%%%%%%%%%%%%%%%%%%%%%%%%%%
\iffalse

\resheading{\textbf{hobbies}}\\[\lsep]
\setlength{\unitlength}{5cm}
\put(0.11,-0.04){\line(1,0){3.6}}\\[+0.2cm]
\hspace*{0.55cm}Volleyball | Gym | Foosball | Sudoku
\\
\resheading{\textbf{Objective}}\\[\lsep]
\setlength{\unitlength}{5cm}
\put(0.11,-0.04){\line(1,0){3.6}}\\[+0.2cm]
\hspace*{0.55cm}To leverage my VLSI design skills and problem solving abilities to work on challenging problems and contribute \hspace*{0.55cm}towards the development of society.

\fi
%% Block comment %%%%%%%%%%%%%%%%%%%%%%%%%%%%%%%%%%%%%%%%

\end{document}
