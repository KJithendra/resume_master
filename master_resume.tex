\documentclass[a4paper,11pt]{article}

\usepackage{times} % Font Package
\usepackage[top=0.5in, bottom=0.5in, left=0.35in, right=0.65in]{geometry}
\usepackage{graphicx} %Use of photos
\usepackage{multicol}

\usepackage{array}

\usepackage[T1]{fontenc}
\usepackage[utf8]{inputenc} % input encoding

\usepackage{nopageno}

\setlength{\tabcolsep}{0in}
\newcommand{\isep}{-2 pt}
\newcommand{\lsep}{-0.5cm}
\newcommand{\psep}{-0.6cm}
\newcommand{\spsep}{-0.75cm}

\newcommand{\tfspace}{\hspace{4.5mm}}

\newcommand{\resheading}[1]{{\large {\begin{minipage}{1\textwidth}{\uppercase{ \textbf{#1}}}\end{minipage}}}}

\renewcommand{\labelitemii}{$\circ$}
%\newcommand{\resheading}[1]{{\Large {\begin{minipage}{0.975\textwidth}{\scshape{\textbf{#1 \vphantom{p\^{E}}}}}\end{minipage}}}}

\usepackage{hyperref}
\hypersetup{
    colorlinks=true,
    linkcolor=cyan,
    filecolor=magenta,      
    urlcolor=blue,
}


\begin{document}

%Personal Details and Institute logo
\begin{minipage}{0.70\linewidth}
\LARGE \uppercase {\textbf{K Jithendra}}\\
\Large \uppercase{{RTL Design Engineer, Intel Corporation}}\\\\
\Large {IIT Madras, NIT Puducherry, Ex-Mphasis}\\
\Large {+91 9489551597 | jithendra.kothakota@gmail.com}\\
\Large {\href{https://www.linkedin.com/in/kjithendra/}{https://www.linkedin.com/in/kjithendra/}}\\
\end{minipage}
\begin{minipage}{0.27\linewidth}
\hfill
\includegraphics[width=30mm]{jithendra_k.png}
\end{minipage}\\

\resheading{\textbf{Professional Experience(3 years and 8 months)}}\\[\lsep]
\setlength{\unitlength}{5cm}
\put(0.11,-0.04){\line(1,0){3.6}}\\[-0.6cm]
\begin{itemize}
	\item RTL Design Engineer at Intel Corporation(Power Management(PM) Unit)\hfill \emph{February 2022 - Present} \\[\spsep]
		\begin{itemize} \itemsep \isep
			\item Owned PM Unit's Dispatcher module that sequences and sends power related work points to various domains and partitions in the SOC. Implemented DVFS handling with firmware override capability to support post silicon fixes for unseen bugs in BattleMage GPU. Fixed bugs and supported validation team till the Tape in.
			\item Implemented a central repository for the PM Unit git repository from which multiple PM Units can be generated. Made necessary changes required in the RTL design flow, collaborated with Tools and flows team to make the necessary changes in the tools used. Demonstrated POC of it, made the incremental changes based on the feedback received and included in the latest project to speed up the design time-line of the project.
		\end{itemize}

	\item Power Management Architect at Intel Corporation\hfill \emph{July 2020 - February 2022} \\[\spsep]
		\begin{itemize} \itemsep \isep
			% Write the important points with quantitative results.
			\item Created High level Architecture specification(HAS) for Arbitration logic between Xtensa Micro controller and TAP logic such that the requests from the micro controller are prioritized to maximize overall performance of the micro controller.
			\item Created HAS for SVID(Serial voltage ID) protocol related power management that is used by the firmware team to transition its source code base from Foxton assembly code to C++.
			\item Automated various stages of HBM Emulation runs in Ponte Vechio Server GPU like preprocessing input files, launching emulation runs for various tests at once in parallel, generating the statistics such as bandwidth, latency and transactions by running the corresponding scripts and generating summaries of the finished emulation runs into spreadsheets and thereby reduced corresponding time required to do these tasks by  approximately 10 times. Debugged and and helped to improve HBM performance by running various scenarios of work loads.		
			\item Modelled Adaptive clock modulation and PID loop in python that can be used to find the best parameters to generate optimal performance.
		\end{itemize}
	
	\item FPGA Desing Engineer at Indigenous 5G Testbed, IIT Madras \hfill May 2019 - June 2020 \\[\spsep]
	% \emph{Indigenous 5G Testbed is } \\[\spsep]
	\begin{itemize} \itemsep \isep
		% \item Design and implementation of 5G technologies for wireless systems and networks on integrated multi-core programmable SoCs for maximal performance. 
		\item Designed and Implemented Channel Decoder module for the Uplink Receiver on Xilinx Zynq UltraScale+ RFSoC.
	\item worked on end to end VLSI design flow for Channel Decoder module such as module design using HLS, verification, synthesis to place and route, testing on the Xilinx Zync FPGA.
	\item Integrated this module into the Uplink Receiver and supported the integration team during the revisions of the receiver.
	\end{itemize}

	\item Associate Software Engineer in Mphasis Pvt. Ltd. \hfill \emph{July 2018 - July 2019}  \\[\spsep]
		\begin{itemize} \itemsep \isep
		\item Worked as a team member in a project to fix vulnerabilities present in Centive, an Incentive Application based on Java Applets.
%		\item Proposed solutions were accepted by the higher members of the team and the development environment was set up.
		\item Developed a bank web application that can be used by bank employees to carryout transactions \& process customer needs and by customers to use internet banking facilities, as a part of full stack web development training in Java programming language.
		\end{itemize}
\end{itemize}

\pagebreak
\resheading{\textbf{Education} }\\[\lsep]
\setlength{\unitlength}{5cm}
\put(0.11,-0.04){\line(1,0){3.6}}\\[-0.6cm]
\begin{table}[h!]
\setlength{\tabcolsep}{3.5pt}
\begin{tabular}{llll}
%\setlength{\unitlength}{5cm}
%\put(0.11,0.1){\line(1,0){3.6}}\\[-0.6cm] 
\tfspace\textbf{Program} & \textbf{Institution} & \textbf{CGPA/\%}           & \textbf{Completion Year}  \\ 
\setlength{\unitlength}{5cm}
\put(0.11,0.1){\line(1,0){3.6}}\\[-0.45cm] 
\tfspace M.Tech. (Microelectronics and VLSI Design)  & IIT Madras  & 8.85 & 2020 \\\\[-0.4cm]
\tfspace B.Tech. (Electronics and Comm. Engg.) & NIT Puducherry & 8.64 & 2017\hfill \\\\[-0.4cm]
\tfspace Intermediate & Sri Vidya Vikas Jr. College Chittoor & 97.7 & 2013\hfill \\\\[-0.4cm]
\tfspace SSC & Jawahar Navodaya Vidyalaya Chittoor & 8.4 & 2011\hfill \\\\ [-0.4cm]
\setlength{\unitlength}{5cm}
\put(0.11,0.1){\line(1,0){3.6}}\\[-0.75cm]
\end{tabular}
\end{table}

\resheading{\textbf{Technical Skills}}\\[\lsep]
\setlength{\unitlength}{5cm}
\put(0.11,-0.04){\line(1,0){3.6}}\\[-0.6cm]
\begin{itemize} \itemsep \isep
  \item \textbf{Programming languages} : Python, Perl(intermediate), tcsh scripting, Java(intermediate), markdown, C, C++, \\
  \hspace*{43mm} TCL(novice). \\[-0.55cm]
  \item \textbf{HDL}\hspace{33mm}: Verilog, System Verilog. \\[-0.55cm].

  \item \textbf{Software packages}\hspace{11mm}: Verdi, Spyglass, GIT, Vivado, Vivado HLS, Cachegrind, LTSpice, Electric, Eclipse, \\
  \hspace*{43mm} MATLAB, CACTI, Ramulator(novice).
\end{itemize}

\resheading{\textbf{Scholastic Achievements} }\\[\lsep]
\setlength{\unitlength}{5cm}
\put(0.11,-0.04){\line(1,0){3.6}}\\[-0.6cm] 
\begin{itemize}\itemsep \isep
	\item Attended India Mobile Congress(IMC), October 2019 where we exhibited Indigenous 5G Testbed Project.\hfill %Team member focussed in the design, implemention and integration of modules present in uplink receiver of physical layer in 5G cellular network.
	\item Secured All India Rank of 347 in Gate 2018 - Electronics and Communication Engineering.\hfill
	\item Participated in national round of INDO-US Robo League 2015 in Line Following event, conducted by Technophilia Systems and Robotics \& Computer Applications Institute of USA, held at IIT Bombay during AAVRITI 2015.
	\item Recipient of Central Sector Scheme of Scholarship for College and University students for the year 2013-14 and received the scholarship for 4 consecutive years from 2013 to 2017.\hfill
	\item Secured All India Rank of 14475 in JEE-MAIN 2013.\hfill
	\item Secured rank of 2724 in EAMCET 2013.\hfill
\end{itemize}

\pagebreak
\resheading{\textbf{Backup} }\\[\lsep]
\setlength{\unitlength}{5cm}
\put(0.11,-0.04){\line(1,0){3.6}}\\\\\\[-0.6cm]

\resheading{\textbf{Key Projects} }\\[\lsep]
\setlength{\unitlength}{5cm}
\put(0.11,-0.04){\line(1,0){3.6}}\\[-0.6cm]
\begin{itemize}
%% Project description template
%	\item \textbf{<Project name> \hfill <date> - <date>} \\
%	\emph{<course name>  \hfill <Skills used>} \\[\spsep]
%	\begin{itemize} \itemsep \isep
%		\item  
%		\item
%	\end{itemize}

	\item \textbf{Scale Sim \hfill <date> - <date>} \\
	\emph{SysDL  \hfill Python, Pytorch} \\[\spsep]
	\begin{itemize} \itemsep \isep
		\item  
		\item
	\end{itemize}
	 
	 \item \textbf{Hardware accelerator for Handwritten digit recognition using MNIST database \hfill January - May 2019} \\
	\emph{Mapping Signal Processing Algorithms to DSP Architectures  \hfill C++, Vivado, Vivado HLS} \\[\spsep]
	\begin{itemize} \itemsep \isep
	\item Implemented hardware accelerator for trained feed forward neural network model from KANN library on Xilinx Zynq-7000 SoC, to speed up the classification of handwritten digit images from MNIST database.
%		\item Implemented a hardware accelerator to speed up the classification of the handwritten digit images from MNIST dataset.
%		\item Already trained multi layer perceptron feed forward neural network model from KANN Library is used to classify the images. This model consists of two fully connected layers and two non linearity layers.
		\item Achieved a speed improvement by a factor of (1.8).
	\end{itemize}
	
	\item \textbf{In memory compute engine for Handwritten digit recognition using MNIST database  \hfill January - May 2019} \\
	\emph{Advanced Topics in VLSI \hfill  Electric, LTSpice, MATLAB} \\[\spsep]
	\begin{itemize} \itemsep \isep
		\item Designed 4$\times$2 array of 8T SRAM Cells and associated peripheral circuitry that computes multiply and accumulate(MAC) operations for fully connected layer of trained neural network model from KANN library.
%		\item Stored the weights into the cells of the array. In compute operation, the array computes the MAC operation between input data and stored weights, then sends the result through the data bus.

	\end{itemize}
	
	\item \textbf{8 bit carry save multiplier with single stage pipeline \hfill July - November 2018} \\
	\emph{Digital IC Design \hfill  Electric, LTSpice} \\[\spsep]
	\begin{itemize} \itemsep \isep
		\item Designed schematic and layout of signed 8 bit carry save multiplier and then RC extracted netlist of Layout is used for simulations.
		\item Achieved 93\% improvement in the maximum frequency of operation of the pipelined multiplier when compared with the unpipelined multiplier.
%		\item Designed an 8 bit Carry save multiplier in transistor level schematic and layout of the same in 22nm CMOS technology, to increase the maximum frequency of operation and throughput.
%		\item Achieved a 64\% improvement in the maximum frequency of operation of the pipelined multiplier when compared with the unpipelined multiplier.
%		\item The use of Carry look ahead adder instead of ripple carry adder in the unpipelined multiplier design improved the maximum frequency of operation by 34\%.
	\end{itemize}
	
	\item \textbf{Performance evaluation and implementation of SVM Classifier for speech emotion recognition using Berlin database \hfill January - April 2017} \\
	\emph{B. Tech Project  \hfill  MATLAB} \\[\spsep]
	\begin{itemize} \itemsep \isep
		\item Trained SVM classifier using training data of 4-dimensional feature vectors. The resultant hyperplane was used as a decision boundary to classify the test data.%that separates the trainining data to a maximum possible extent.
		\item Achieved an average classification accuracy of 72.04\%.
	\end{itemize}
	
%	\item \textbf{Web application for a bank  \hfill October 2017}  \\
%	\emph{Full Stack Web Development Training.\hfill Java, Spring MVC 4.0} \\[\spsep]
%	\begin{itemize} \itemsep \isep
%		\item Developed a bank web application that can be used by bank employees to carryout transactions \& process customer needs and by customers to use internet banking facilities.
%	\end{itemize}
\end{itemize}
\


\resheading{\textbf{Course Work}}\\[\lsep]
\setlength{\unitlength}{5cm}
\put(0.11,-0.04){\line(1,0){3.6}}\\[-0.6cm]
\\\\[-0.1cm]
\begin{minipage}[t]{10cm}
	\begin{itemize}\itemsep \isep
		\item Computer architecture
		\item Mapping signal processing algorithms to DSP architectures
		\item Digital IC design
		\item Digital system testing and testable design
		\item Advanced topics in VLSI	
\end{itemize}
\end{minipage}
\begin{minipage}[t]{8cm}
	\begin{itemize}\itemsep \isep
		\item Pattern recognition\hfill
		\item Networks and protocols\hfill
		\item VLSI technology
		\item Semiconductor device modelling
		\item Microprocessors and microcontrollers\hfill		
\end{itemize}
\end{minipage}\\\\[-0.3cm]



\resheading{\textbf{Laboratories}}\\[\lsep]
\setlength{\unitlength}{5cm}
\put(0.11,-0.04){\line(1,0){3.6}}\\[-0.6cm]
\\\\[-0.1cm]
\begin{minipage}[t]{10cm}
	\begin{itemize}\itemsep \isep
		\item VLSI design laboratory\hfill
		\item Microprocessors and microcontrollers laboratory\hfill
\end{itemize}
\end{minipage}
\begin{minipage}[t]{8cm}
	\begin{itemize}\itemsep \isep
		\item Electronic circuits laboratory\hfill 
		\item Digital Electronics laboratory\hfill		
\end{itemize}
\end{minipage}\\\\[-0.3cm]

\resheading{\textbf{Positions of Responsibility}}\\[\lsep]
\setlength{\unitlength}{5cm}
\put(0.11,-0.04){\line(1,0){3.6}}\\[-0.6cm]
\begin{itemize} \itemsep  \isep
	
	\item Teaching assistant - Computer organization course \hfill \emph{August - November 2019} \\[\spsep]
	\begin{itemize} \itemsep \isep
		\item Mentored and evaluated students in the design and implementation of a pipelined CPU that supports RV32I Base Instruction Set of RISC-V ISA.
		\item Assisted in creating the testbenches for assignments.
		
	\end{itemize}
	
	\item Teaching assistant - NPTEL - Mapping signal processing algorithms to architectures \hfill \emph{August - November 2019} \\[\spsep]
	\begin{itemize} \itemsep \isep
		\item Resolved doubts that were asked in the discussion forum of this online course.
	\end{itemize}
	
	\item Teaching assistant - Digital systems and Lab \hfill \emph{January - May 2019} \\[\spsep]
	\begin{itemize} \itemsep \isep
		\item Coordinated weekly lab sessions and assisted students during these lab sessions. 
		\item Evaluated lab assignments and exam papers.
	\end{itemize}
	
	%\item Captain of Institute volleyball team during All India Inter NIT sports meet 2016-17 organized by NIT surathkal.
	
	%\item Mentor to NPTEL Course named Mapping signal Processing algorithms to DSP Architectures. \hfill \emph{(August-November 2019)}
\end{itemize}

%\resheading{\textbf{Co-Curricular Activities}}\\[\lsep]
%\begin{itemize} \itemsep  \isep
%\item Participated in Line following event at the national round of Indo-US robo League 2015 conducted by Technophilia systems and Robotics and Computer Applications Institute of USA, held at IIT bombay during Aavriti 2015 organized by EESA IIT Bombay, on 28th and 29th march 2015.
%\item Accelero-Botix, A workshop conducted by technophilia systems in association with robotics and computer applications institute of USA held at national institute of technology Trichy on 27and 28 feb 2015.
%\item Hand gesture controlled robotics workshop organized.
% by Pragyan, The international techno- management festival of NIT tiruchirapalli held from 26 feb to 1 march 2015.
%\item Arduinotics workshop conducted by DigimindIndia pvt ltd. at Tarang 14, the annual technical symposium of ece dept held at nit pducherry on 28 feb 2014.
%\item Android application developement workshop conducted during knosys 14 on 31 jan 2014 at NIT puducherry.
%\end{itemize}

\resheading{\textbf{Extra Curricular Activities}}\\[\lsep]
\setlength{\unitlength}{5cm}
\put(0.11,-0.04){\line(1,0){3.6}}\\[-0.6cm]
\begin{itemize} \itemsep  \isep
	\item Achieved second position in Volleyball Tournament during schroeter 2018-19, held at IIT madras.\hfill
	\item Achieved second position in Chess competition in Annual Sports meet 2016-17, held at NIT Puducherry.\hfill
	\item Participated and completed Sports for mental health run (5KM running competition) conducted by Shaastra Sports Tech Summit and Decathlon on October 28, 2018 in IIT Madras.\hfill
%	
%	\item Achieved first position in Freshie volleyball tournament during 2018-19 held at IIT Madras.
%	\item Achieved First position in Carroms competition in Annual Sports meet 2016-17.
%	
%	\item Participated in Volleyball Tournament during All India Inter NIT sports meet 2016-17 organized by NIT surathkal.
%	\item Achieved second position in Chess competition at the annual sports meet 2015 of NIT Puducherry.
%	\item Achieved second position in Volleyball tournament at the annual sports meet 2015 of NIT Puducherry.
%	\item Participated in Volleyball Tournament during All India Inter NIT sports meet 2013-14 organized by NIT Warangal.
%	\item Achieved first position in Kabaddi competition held in connection with college day celebration for the year 2011-12 by Sri Vidya vikas junior college, chittoor.

	
\end{itemize}

\resheading{\textbf{Objective}}\\[\lsep]
\setlength{\unitlength}{5cm}
\put(0.11,-0.04){\line(1,0){3.6}}\\[+0.2cm]
\hspace*{0.55cm}To leverage my VLSI design skills and problem solving abilities to work on challenging problems and contribute \hspace*{0.55cm}towards the development of society.

\end{document}
